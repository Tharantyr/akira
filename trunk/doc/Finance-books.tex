\documentclass[8pt,a4paper]{article}
\begin{document}
\title{Finance Study Notes}
\author{Eric Chung}
\date{17, July, 2007}
\maketitle
\section{Analysing and Interpreting the Yield Curve}

\subsection{Definition of Yield}
The yield on any investment is the interest rate that will make the present value of the cash flows from the investment equal to the initial cost(price) of the investment.
\[P=\sum_{n=1}^N \frac{C_n}{(1+r)^n}\]
where
\begin{description}
	\item [$C_n$] is the cash flow in the year $n$;
	\item [$P$] is the price of the investment;
	\item [$n$] is the number of years.
\end{description}

\subsection{Current Yield}
Also known as \textit{flat yield, interest yeild} or \textit{running yield}. It's the simplest measure of the yield on a bond.
\[rc=\frac{C}{P}\times 100\]
Current yield ignores any capital gain or loss that might arise from holding and trading a bond and doesn't consider the time value of money.

\subsection{Simple Yield to Maturity}
\[rs=\frac{C}{P}+\frac{100 - P}{nP}\]

\subsection{Yield to Maturity}

The \textit{yield to maturity} (YTM) or \textit{gross redemption yield} is equivalent to the \textit{internal rate of return} on the bond.
\[P_d=
		\frac{C}{(1+rm)^1} +
		\frac{C}{(1+rm)^2} +		
		\frac{C}{(1+rm)^3} +...+
		\frac{C}{(1+rm)^n} +		
		\frac{M}{(1+rm)}		
\] 
where
\begin{description}
	\item[$P_d$] is the bond dirty price;
	\item[$M$] is the par or redemption payment (100);
	\item[$n$] is the number of interest periods;
	\item[$C$] is the coupon rate;
	\item[$rm$] is the annual yield to maturity (the YTM).
\end{description}

The YTM equation use $rm$ to discount a bond's cash flows back to the next coupon payment and then discount the value at that date back to the date of calculation. In other words $rm$ is the \textit{internal rate of return}(IRR) that equates the value of the discounted cash flows on the bond to the current dirty price of the bond(at the current date).

If the settlement date falls in between coupon dates, the formula is adjusted to allow for the uneven interest period:
\[P_d=
		\frac{C}{(1+rm)^w} +
		\frac{C}{(1+rm)^{1+w}} +
		\frac{C}{(1+rm)^{2+w}} +...+
		\frac{C}{(1+rm)^{n-1+w}} +
		\frac{M}{(1+rm)^{n-1+w}}
\]
where
w is 
$\frac{\mbox{number of days between the settlement date and the next coupon date}}{\mbox{number of days in the interest period}}$
and n is the number of coupon payments remaining in the life of bond. The formula can be shorten to:
\[
P_d=\sum_{n=1}^N \frac{C}{(1+rm)^{n-1+w}} + \frac{M}{(1+rm)^{n-1+w}}
\]

\subsection{Yield on a zero-coupon bond}
Zero-coupon bond has only one cash flow, the redemption payment on maturity. The payment will be par. So a zero-coupon bond is sold at a discount to par and trades at a discount during its life.
\[
	P=\frac{C}{(1+rm)^n}
\]

\subsection{Modifying Bond Yields}
\subsubsection{Very short-dated bonds}
The usual convention in the markets is to adjust bond yeilds using money market convention by discounting the cash flow at a simple rate of interest instead of a compound rate.
\[
	m=(\frac{\mbox{final cash flow}}{P_d} - 1)\times \frac{B}{\mbox{day to maturity}}
\]
where B is the day-base count for the bond.
\subsubsection{Money market yields}
From time to time to compare with other money market instruments, we need to convert the bond yield into money market yield.
\[
	P_d=\frac{M}{(1+(\frac{r_{me}}{n}\times t))}\times(\frac{C}{n}\times \frac{1-\frac{1}{(1+\frac{r_{me}}{n}\times \frac{365}{360})^N}}{1-\frac{1}{(1+\frac{r_{me}}{n}\times \frac{365}{360})}} + \frac{1}{1-\frac{1}{(1+\frac{r_{me}}{n}\times \frac{365}{360})^{N-1}}})
\]
where:\\
\begin{tabbing}
$r_{me}$ \=\= is the bond money-market yield\\
$t$ \>\> is the fraction of the bond coupon period, on a money market basis.
\end{tabbing}

\subsubsection{Moosmuller yield}
Similar to money market yield calculation, because it discounts the next coupon from the settlement date using simple rather than compounding interest. The day-count basis remains the one used in the bond markets.
\[
P_d=\frac{M}{(1+(\frac{r_m}{n}\times t))}\times (\frac{C}{n}\times \frac{1-\frac{1}{(1+\frac{r_m}{n})^N}}{1-\frac{1}{{1+\frac{r_m}{n}}}} + \frac{1}{1-\frac{1}{(1+\frac{r_m}{n})^{N-1}}})
\]
where $r_m$ is the Moosmuller yield.
\end{document}