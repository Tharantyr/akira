\documentclass[8pt,a4paper]{article}
\begin{document}
\title{Finance Study Notes}
\author{Eric Chung}
\date{17, July, 2007}
\maketitle
\section{Analysing and Interpreting the Yield Curve}

\subsection{Definition of Yield}
The yield on any investment is the interest rate that will make the present value of the cash flows from the investment equal to the initial cost(price) of the investment.
\[P=\sum_{n=1}^N \frac{C_n}{(1+r)^n}\]
where
\begin{tabbing}
	$C_n$ \= is the cash flow in the year $n$;\\
	$P$ \> is the price of the investment;\\
	$n$ \> is the number of years.
\end{tabbing}

\subsection{Current Yield}
Also known as \textit{flat yield, interest yeild} or \textit{running yield}. It's the simplest measure of the yield on a bond.
\[rc=\frac{C}{P}\times 100\]
Current yield ignores any capital gain or loss that might arise from holding and trading a bond and doesn't consider the time value of money.

\subsection{Simple Yield to Maturity}
\[rs=\frac{C}{P}+\frac{100 - P}{nP}\]

\subsection{Yield to Maturity}

The \textit{yield to maturity} (YTM) or \textit{gross redemption yield} is equivalent to the \textit{internal rate of return} on the bond.
\[P_d=
		\frac{C}{(1+rm)^1} +
		\frac{C}{(1+rm)^2} +		
		\frac{C}{(1+rm)^3} +...+
		\frac{C}{(1+rm)^n} +		
		\frac{M}{(1+rm)}		
\] 
where
\begin{tabbing}
	$rm$ \= \= is the annual yield to maturity (the YTM).\\
	$P_d$ \> \> is the bond dirty price;\\
	$M$ \> \> is the par or redemption payment (100);\\
	$n$ \> \> is the number of interest periods;\\
	$C$ \> \> is the coupon rate;
\end{tabbing}

The YTM equation use $rm$ to discount a bond's cash flows back to the next coupon payment and then discount the value at that date back to the date of calculation. In other words $rm$ is the \textit{internal rate of return}(IRR) that equates the value of the discounted cash flows on the bond to the current dirty price of the bond(at the current date).

If the settlement date falls in between coupon dates, the formula is adjusted to allow for the uneven interest period:
\[P_d=
		\frac{C}{(1+rm)^w} +
		\frac{C}{(1+rm)^{1+w}} +
		\frac{C}{(1+rm)^{2+w}} +...+
		\frac{C}{(1+rm)^{n-1+w}} +
		\frac{M}{(1+rm)^{n-1+w}}
\]
where
w is 
$\frac{\mbox{number of days between the settlement date and the next coupon date}}{\mbox{number of days in the interest period}}$
and n is the number of coupon payments remaining in the life of bond. The formula can be shorten to:
\[
P_d=\sum_{n=1}^N \frac{C}{(1+rm)^{n-1+w}} + \frac{M}{(1+rm)^{n-1+w}}
\]

\subsection{Yield on a zero-coupon bond}
Zero-coupon bond has only one cash flow, the redemption payment on maturity. The payment will be par. So a zero-coupon bond is sold at a discount to par and trades at a discount during its life.
\[
	P=\frac{C}{(1+rm)^n}
\]

\subsection{Modifying Bond Yields}
\subsubsection{Very short-dated bonds}
The usual convention in the markets is to adjust bond yeilds using money market convention by discounting the cash flow at a simple rate of interest instead of a compound rate.
\[
	m=(\frac{\mbox{final cash flow}}{P_d} - 1)\times \frac{B}{\mbox{day to maturity}}
\]
where B is the day-base count for the bond.
\subsubsection{Money market yields}
From time to time to compare with other money market instruments, we need to convert the bond yield into money market yield.
\[
	P_d=\frac{M}{(1+(\frac{r_{me}}{n}\times t))}\times(\frac{C}{n}\times \frac{1-\frac{1}{(1+\frac{r_{me}}{n}\times \frac{365}{360})^N}}{1-\frac{1}{(1+\frac{r_{me}}{n}\times \frac{365}{360})}} + \frac{1}{1-\frac{1}{(1+\frac{r_{me}}{n}\times \frac{365}{360})^{N-1}}})
\]
where:\\
\begin{tabbing}
$r_{me}$ \=\= is the bond money-market yield\\
$t$ \>\> is the fraction of the bond coupon period, on a money market basis.
\end{tabbing}

\subsubsection{Moosmuller yield}
Similar to money market yield calculation, because it discounts the next coupon from the settlement date using simple rather than compounding interest. The day-count basis remains the one used in the bond markets.
\[
P_d=\frac{M}{(1+(\frac{r_m}{n}\times t))}\times (\frac{C}{n}\times \frac{1-\frac{1}{(1+\frac{r_m}{n})^N}}{1-\frac{1}{{1+\frac{r_m}{n}}}} + \frac{1}{1-\frac{1}{(1+\frac{r_m}{n})^{N-1}}})
\]
where $r_m$ is the Moosmuller yield.

\subsection{Converting Bond Yields}
\subsubsection{Discounting and coupon frequency}
Semi-annual discounting of semi-annaual payments:
\[
P_d=\frac{C/2}{(1+\frac{1}{2}rm)} +
		\frac{C/2}{(1+\frac{1}{2}rm)^2} +
		\frac{C/2}{(1+\frac{1}{2}rm)^3} +...+
		\frac{C/2}{(1+\frac{1}{2}rm)^N} +
		\frac{M}{(1+\frac{1}{2}rm)^{2N}}
\]
Annual discounting of annual payments:
\[
P_d=\frac{C}{(1+\frac{1}{2}rm)} +
		\frac{C}{(1+\frac{1}{2}rm)^2} +
		\frac{C}{(1+\frac{1}{2}rm)^3} +...+
		\frac{C}{(1+\frac{1}{2}rm)^N} +
		\frac{M}{(1+\frac{1}{2}rm)^N}
\]
Semi-annual discounting of annual payments:
\[
P_d=\frac{C}{(1+\frac{1}{2}rm)^2} +
		\frac{C}{(1+\frac{1}{2}rm)^4} +
		\frac{C}{(1+\frac{1}{2}rm)^6} +...+
		\frac{C}{(1+\frac{1}{2}rm)^{2N}} +
		\frac{M}{(1+\frac{1}{2}rm)^{2N}}
\]
Annual discounting of semi-annual payments:
\[
P_d=\frac{C/2}{(1+\frac{1}{2}rm)^2} +
		\frac{C/2}{(1+\frac{1}{2}rm)^4} +
		\frac{C/2}{(1+\frac{1}{2}rm)^6} +...+
		\frac{C/2}{(1+\frac{1}{2}rm)^N} +
		\frac{M}{(1+\frac{1}{2}rm)^N}
\]

\subsubsection{Converting yields}
Note that doubling a semi-annual yield figure will NOT give us the annualized equivalent.
The general conversion expression is given:\\
$rm_a=(1+\mbox{interest rate})^m - 1$\\
where $m$ is the number of coupon payments per year.\\
We can convert between annual and semi-annual compounded yields using the following:\\
$rm_a=((1+\frac{1}{2}rm_s)^2 - 1)$\\
$rm_s=((1+rm_a)^{\frac{1}{2}} - 1)\times 2$

\subsubsection{Assumption of the Redemption Yield Calculation}
The disadvantage of YTM computation:
\begin{itemize}
	\item It assumes each coupon payment is re-invested at the same rate rm.
	\item YTM measure fails to measure if investor do not hold to maturity.
\end{itemize}

\subsubsection{Holding-period Yield}
The return generated by a bond holding is a function of the purchase price and the sale price on disposal, in addition to the coupons received. It's also called \textit{reinvestment yield}. It's the average yield realized during the holding period, taking into account changes in the \textit{rollover rate}(the interest rate at which coupon payments are reinvested). Since the rollover interest rate will fluctuate in between coupon dates there is a chance that it will be below the bond's redemption yield. The risk that the rollover rate is less than the yield to maturity is known as reinvestment risk.
\[
	P_d(1+\frac{1}{2}rh)^{2N}=(\frac{C}{2})(1+\frac{1}{2}r_1)^{2N-1} +
		(\frac{C}{2})(1+\frac{1}{2}r_2)^{2N-2} +...+(\frac{C}{2}) + P_1
\]
\[
	rh=((\frac{(\frac{C}{2})(1+\frac{1}{2}r_1)^{2N-1}+...+(\frac{C}{2}) + P_1}{P_d})^{\frac{1}{2N}} - 1)\times 2
\]

\subsection{Bonds with Embedded Options}
\subsubsection{Callable Bond}
A callable bond contains a provision that allows the borrower to redeem all or part of the issue before the stated maturity date. The issuer generally calls a bond when falling interest rates or an improvement in their borrowing status makes it worthwhile to cancel existing and replace them at a lower rate of interest. It effectively put a cap on the value of the bond. \textit{yield-to-call} is the discount rate that equates the NPV of the bonds cash flows.
\[
	P_d=\frac{C}{(1+r_ca)} +
			\frac{C}{(1+r_ca)^2} +
			\frac{C}{(1+r_ca)^3} + ... +
			\frac{C}{(1+r_ca)^{Nc}} +
			\frac{P_c}{(1+r_ca)^{Nc}}
\]
where:
\begin{tabbing}
$r_{ca}$ 	\= is the yield-to-call;\\
$N_c$		\> is the number of years to the assumed call date;\\
$P_c$		\> is the call price at the assumed call date (par, or as given in call schedule).
\end{tabbing}
The above equation can be rewritten as:
\[
	P_d=\sum_{n=1}^{Nc}\frac{C}{(1+r_{ca})^n} + \frac{P_c}{(1+r_{ca})^{Nc}}
\]

\subsubsection{Putable Bond}
A putable bond grants the bondholder the right to sell the bond back to the issuer, usually in accordance with specified terms and conditions. The price/yield formula used to calculate yield-to-put is identical to the formula for yield-to-call.
\[
	P_d=\sum_{n=1}^{Nc} \frac{C}{(1+r_p)^n} + \frac{P_p}{(1+r_p)^N}
\]
where
\begin{tabbing}
$r_p$ \= is the yield-to-put;\\
$N$ \> is the number of years to the assumed put date;\\
$P_p$ \> is the put price on the assumed
\end{tabbing}

\subsubsection{Yield to average life}

\end{document}